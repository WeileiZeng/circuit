% Weilei Zeng
% A circuit to measure XXX XXX in shor's code
% There are three versions of this circuit 1>standard one 2>simplified one 3>more simplified
% Here shows 1>the standard one
% 
%   qubit	   q0
%   qubit	   q1
%   qubit	   q2
%   qubit	   q3
%   qubit	   q4
%   qubit	   q5
%   qubit	   A0,0
% 
%   H	   q0
%   H	   q1
%   H	   q2
%   H	   q3
%   H	   q4
%   H	   q5
%   cnot	   q0,A0
%   cnot	   q1,A0
%   cnot	   q2,A0
%   cnot	   q3,A0
%   cnot	   q4,A0
%   cnot	   q5,A0
%   nop	   q0
%   nop	   q0
%   nop	   q0
%   nop	   q0
%   nop	   q0
%   nop	   q1
%   nop	   q1
%   nop	   q1
%   nop	   q1
%   nop	   q2
%   nop	   q2
%   nop	   q2
%   nop	   q3
%   nop	   q3
%   nop	   q4
%   H	   q0
%   H	   q1
%   H	   q2
%   H	   q3
%   H	   q4
%   H	   q5
%   measure  A0
%  Time 01:
%    Gate 00 H(q0)
%    Gate 01 H(q1)
%    Gate 02 H(q2)
%    Gate 03 H(q3)
%    Gate 04 H(q4)
%    Gate 05 H(q5)
%  Time 02:
%    Gate 06 cnot(q0,A0)
%  Time 03:
%    Gate 07 cnot(q1,A0)
%    Gate 12 nop(q0)
%  Time 04:
%    Gate 08 cnot(q2,A0)
%    Gate 13 nop(q0)
%    Gate 17 nop(q1)
%  Time 05:
%    Gate 09 cnot(q3,A0)
%    Gate 14 nop(q0)
%    Gate 18 nop(q1)
%    Gate 21 nop(q2)
%  Time 06:
%    Gate 10 cnot(q4,A0)
%    Gate 15 nop(q0)
%    Gate 19 nop(q1)
%    Gate 22 nop(q2)
%    Gate 24 nop(q3)
%  Time 07:
%    Gate 11 cnot(q5,A0)
%    Gate 16 nop(q0)
%    Gate 20 nop(q1)
%    Gate 23 nop(q2)
%    Gate 25 nop(q3)
%    Gate 26 nop(q4)
%  Time 08:
%    Gate 27 H(q0)
%    Gate 28 H(q1)
%    Gate 29 H(q2)
%    Gate 30 H(q3)
%    Gate 31 H(q4)
%    Gate 32 H(q5)
%    Gate 33 measure(A0)

% Qubit circuit matrix:
%
% q0: gAxA, gBxA, gCxA, gDxA, gExA, gFxA, gGxA, gHxA, n  
% q1: gAxB, n  , gCxB, gDxB, gExB, gFxB, gGxB, gHxB, n  
% q2: gAxC, n  , n  , gDxC, gExC, gFxC, gGxC, gHxC, n  
% q3: gAxD, n  , n  , n  , gExD, gFxD, gGxD, gHxD, n  
% q4: gAxE, n  , n  , n  , n  , gFxE, gGxE, gHxE, n  
% q5: gAxF, n  , n  , n  , n  , n  , gGxF, gHxF, n  
% A0: n  , gBxG, gCxG, gDxG, gExG, gFxG, gGxG, gHxG, N  

\documentclass[11pt]{article}
%
% File:   xyqcirc.tex
% Date:   14-Mar-04
% Author: I. Chuang <ichuang@mit.edu>
%
% Definitions for producing quantum circuits with XYPIC in latex
%
% $Log: xyqcirc.tex,v $
% Revision 1.17  2004/03/25 05:01:23  ike
% discard and slash
%
% Revision 1.16  2004/03/25 04:58:42  ike
% added discard, and variable width dmeter
%
% Revision 1.15  2004/03/24 23:43:33  ike
% \dmeter and \sq
%
% Revision 1.14  2004/03/24 20:29:40  ike
% added \t for swap
%
% Revision 1.13  2004/03/24 17:52:16  ike
% removed \w from \gspace
%
% Revision 1.12  2004/03/24 16:38:34  ike
% added small space before |0> for \z
%
% Revision 1.11  2004/03/24 16:23:11  ike
% added \z
%
% Revision 1.10  2004/03/24 16:19:11  ike
% added multiple qubit operations
%
% Revision 1.9  2004/03/24 03:03:44  ike
% typo
%
% Revision 1.8  2004/03/24 02:50:09  ike
% added qv
%
% Revision 1.7  2004/03/24 00:07:34  ike
% add \m matrix op
%
% Revision 1.6  2004/03/23 23:13:10  ike
% misc
%
% Revision 1.5  2004/03/23 23:12:42  ike
% works now
%
% Revision 1.4  2004/03/23 22:22:34  ike
% ifthen also failes - because xymatrix entries in \save...\restore
%
% Revision 1.3  2004/03/23 21:34:36  ike
% no q/c wire switching
%
% Revision 1.2  2004/03/23 21:25:29  ike
% classical qo quantum wire switching try
%
% Revision 1.1  2004/03/23 21:01:46  ike
% Initial revision
%

%%%%%%%%%%%%%%%%%%%%%%%%%%%%%%%%%%%%%%%%%%%%%%%%%%%%%%%%%%%%%%%%%%%%%%%%%%%%%
% preliminaries

\usepackage{graphicx}

\usepackage[frame,line,arrow,matrix,tips]{xy}	% all that is usually necessary
\CompilePrefix{xygui-}
\makeindex
\pagestyle{empty}

\setlength{\oddsidemargin}{-0.5in}	% 1.25in left margin 
\setlength{\evensidemargin}{-0.5in}	% 1.25in left margin (even pages)

\setlength{\topmargin}{0.0in}		% 1in top margin
\setlength{\textwidth}{16.25in}		% 6.0in text - 1.25in rt
% margin
% the defult textheight is 8.6in. Weilei changed it to show super long
% circuit. Jan 2019
%default testwidth 6.25in
\setlength{\textheight}{28.6in}		% Body ht for 1in margins
\addtolength{\topmargin}{-\headheight}	% No header, so compensate
\addtolength{\topmargin}{-\headsep}	% for header height and separation

\begin{document}

\thispagestyle{empty}

%%%%%%%%%%%%%%%%%%%%%%%%%%%%%%%%%%%%%%%%%%%%%%%%%%%%%%%%%%%%%%%%%%%%%%%%%%%%%
% wires

\def\w{\ar@{-}[l]}
\def\W{\ar@{=}[l]}

%%%%%%%%%%%%%%%%%%%%%%%%%%%%%%%%%%%%%%%%%%%%%%%%%%%%%%%%%%%%%%%%%%%%%%%%%%%%%
% labels

% simple label
\def\A#1{\save []="#1" \restore}

%%%%%%%%%%%%%%%%%%%%%%%%%%%%%%%%%%%%%%%%%%%%%%%%%%%%%%%%%%%%%%%%%%%%%%%%%%%%%
% single qubit operations

\def\op#1{*+[F]{\rule[-0.2ex]{0ex}{2.1ex}#1}}	% operator in box
\def\b{*={\bullet}}
\def\o{*={\oplus}}
\def\t{*={\times}}				% for swap gate
\def\sq{*=<6pt,6pt>[F]{}}			% square, for controlled-phase
\def\m#1{\left[\matrix{#1}\right]}		% matrix shortcut
\def\z{*+[]{\rule[-0.2ex]{0ex}{2.1ex}~|0\>}}	% re-init to |0>
\def\discard{*[]{\rule[-0.2ex]{0.75pt}{2.1ex}~}}	% vertical ``|''
\def\slash{*={/}}				% slash for wire bundles

%%%%%%%%%%%%%%%%%%%%%%%%%%%%%%%%%%%%%%%%%%%%%%%%%%%%%%%%%%%%%%%%%%%%%%%%%%%%%
% nop's

\def\N{*-{}\W}
\def\n{*-{}\w}

%%%%%%%%%%%%%%%%%%%%%%%%%%%%%%%%%%%%%%%%%%%%%%%%%%%%%%%%%%%%%%%%%%%%%%%%%%%%%
% misc definitions

\def\>{\rangle}
\def\<{\langle}
\def\ua{\uparrow}

% measurement box
\def\meter{*+[]{\put(-3,0){\includegraphics[scale=.5]{meter.epsf}}~~~~}%
		\ar@{-}[l]}

%%%%%%%%%%%%%%%%%%%%%%%%%%%%%%%%%%%%%%%%%%%%%%%%%%%%%%%%%%%%%%%%%%%%%%%%%%%%%
% qubit names (and also revert to qubit wires, vs, cbit wires)

\def\q#1{*+{\rule[-0.2ex]{0ex}{2.1ex}|#1\>}}
\def\qv#1#2{*+{\rule[-0.2ex]{0ex}{2.1ex}|#1\>=|#2\>}}
	
%%%%%%%%%%%%%%%%%%%%%%%%%%%%%%%%%%%%%%%%%%%%%%%%%%%%%%%%%%%%%%%%%%%%%%%%%%%%%
% multiple qubit gates

% utulity text box for figuring out width of things
\newbox{\sbox}

% empty space of width determined by the text argument
\def\gspace#1{*+{\rule[-0.2ex]{0ex}{2.1ex}%
	\setbox\sbox=\hbox{$#1$}%
	\hspace*{\wd\sbox}}}
	
% n-qubit operation #1=box label, #2=number of qubits (eg d=2 qubits, ddd=4)
\def\gnqubit#1#2{\gspace{#1}
		 \save [].[#2]!C="qq"*[F]\frm{}\restore
		 \save "qq"*[]{#1} \restore}

% two-qubit operation
\def\gtwo#1{\gnqubit{#1}{d}}

% two-qubit operation
\def\gthree#1{\gnqubit{#1}{dd}}

%%%%%%%%%%%%%%%%%%%%%%%%%%%%%%%%%%%%%%%%%%%%%%%%%%%%%%%%%%%%%%%%%%%%%%%%%%%%%
% ``D'' style measurement gate a-la-cleve, at Michael Nielsen's request

\def\dmeterwide#1#2{*{\xy <0pt,-8pt>;<0pt,8pt> **@{-};
		    <0pt,-8pt>;<#2,-8pt> **@{-} ;
		    <0pt, 8pt>;<#2, 8pt> **@{-} ;
		    <#2,0pt>-<5pt,0pt>*{#1} ;
		    <#2,0pt>*\cir<8pt>{r_l}\endxy}}

\def\dmeter#1{\dmeterwide{#1}{12pt}}

%%%%%%%%%%%%%%%%%%%%%%%%%%%%%%%%%%%%%%%%%%%%%%%%%%%%%%%%%%%%%%%%%%%%%%%%%%%%%


% definitions for the circuit elements

\def\gAxA{\op{H}\w\A{gAxA}}
\def\gAxB{\op{H}\w\A{gAxB}}
\def\gAxC{\op{H}\w\A{gAxC}}
\def\gAxD{\op{H}\w\A{gAxD}}
\def\gAxE{\op{H}\w\A{gAxE}}
\def\gAxF{\op{H}\w\A{gAxF}}
\def\gBxA{\b\w\A{gBxA}}
\def\gBxG{\o\w\A{gBxG}}
\def\gCxB{\b\w\A{gCxB}}
\def\gCxG{\o\w\A{gCxG}}
\def\gDxC{\b\w\A{gDxC}}
\def\gDxG{\o\w\A{gDxG}}
\def\gExD{\b\w\A{gExD}}
\def\gExG{\o\w\A{gExG}}
\def\gFxE{\b\w\A{gFxE}}
\def\gFxG{\o\w\A{gFxG}}
\def\gGxF{\b\w\A{gGxF}}
\def\gGxG{\o\w\A{gGxG}}
\def\gCxA{*-{}\w\A{gCxA}}
\def\gDxA{*-{}\w\A{gDxA}}
\def\gExA{*-{}\w\A{gExA}}
\def\gFxA{*-{}\w\A{gFxA}}
\def\gGxA{*-{}\w\A{gGxA}}
\def\gDxB{*-{}\w\A{gDxB}}
\def\gExB{*-{}\w\A{gExB}}
\def\gFxB{*-{}\w\A{gFxB}}
\def\gGxB{*-{}\w\A{gGxB}}
\def\gExC{*-{}\w\A{gExC}}
\def\gFxC{*-{}\w\A{gFxC}}
\def\gGxC{*-{}\w\A{gGxC}}
\def\gFxD{*-{}\w\A{gFxD}}
\def\gGxD{*-{}\w\A{gGxD}}
\def\gGxE{*-{}\w\A{gGxE}}
\def\gHxA{\op{H}\w\A{gHxA}}
\def\gHxB{\op{H}\w\A{gHxB}}
\def\gHxC{\op{H}\w\A{gHxC}}
\def\gHxD{\op{H}\w\A{gHxD}}
\def\gHxE{\op{H}\w\A{gHxE}}
\def\gHxF{\op{H}\w\A{gHxF}}
\def\gHxG{\meter\w\A{gHxG}}

% definitions for bit labels and initial states

\def\bA{ \q{q_{0}}}
\def\bB{ \q{q_{1}}}
\def\bC{ \q{q_{2}}}
\def\bD{ \q{q_{3}}}
\def\bE{ \q{q_{4}}}
\def\bF{ \q{q_{5}}}
\def\bG{\qv{A_{0}}{0}}

% The quantum circuit as an xymatrix

\xymatrix@R=5pt@C=10pt{
    \bA & \gAxA &\gBxA &\gCxA &\gDxA &\gExA &\gFxA &\gGxA &\gHxA &\n  
\\  \bB & \gAxB &\n   &\gCxB &\gDxB &\gExB &\gFxB &\gGxB &\gHxB &\n  
\\  \bC & \gAxC &\n   &\n   &\gDxC &\gExC &\gFxC &\gGxC &\gHxC &\n  
\\  \bD & \gAxD &\n   &\n   &\n   &\gExD &\gFxD &\gGxD &\gHxD &\n  
\\  \bE & \gAxE &\n   &\n   &\n   &\n   &\gFxE &\gGxE &\gHxE &\n  
\\  \bF & \gAxF &\n   &\n   &\n   &\n   &\n   &\gGxF &\gHxF &\n  
\\  \bG & \n   &\gBxG &\gCxG &\gDxG &\gExG &\gFxG &\gGxG &\gHxG &\N  
%
% Vertical lines and other post-xymatrix latex
%
\ar@{-}"gBxG";"gBxA"
\ar@{-}"gCxG";"gCxB"
\ar@{-}"gDxG";"gDxC"
\ar@{-}"gExG";"gExD"
\ar@{-}"gFxG";"gFxE"
\ar@{-}"gGxG";"gGxF"
}

\end{document}
